\documentclass[a4paper,11pt]{article}
\usepackage[T1]{fontenc}
\usepackage[utf8]{inputenc}
\usepackage{lmodern}
\usepackage[francais]{babel}
\usepackage{amsmath} % math
\usepackage{amssymb} % math
\usepackage{gensymb} % math
\usepackage{graphicx} % images
% \usepackage{qtree}    % dessiner des arbres %% => texlive-humanities
\usepackage{url}
\urlstyle{sf}
\usepackage[usenames]{color}
\usepackage[french]{varioref} % \vpageref
\usepackage[top=2.5cm, bottom=2.5cm, left=2.5cm, right=2.5cm]{geometry}
\definecolor{codeBlue}{rgb}{0,0,1}
\definecolor{webred}{rgb}{0.5,0,0}
\definecolor{codeGreen}{rgb}{0,0.5,0}
\definecolor{codeGrey}{rgb}{0.6,0.6,0.6}
\definecolor{webdarkblue}{rgb}{0,0,0.4}
\definecolor{webgreen}{rgb}{0,0.3,0}
\definecolor{webblue}{rgb}{0,0,0.8}
\definecolor{orange}{rgb}{0.7,0.1,0.1}
\usepackage{caption}
\renewcommand{\familydefault}{\sfdefault}
\usepackage{listings}        % Pour l'insersion de fichiers de codes sources.
\lstset{
      language=Java,
      flexiblecolumns=true,
      numbers=left,
      stepnumber=1,
      numberstyle=\ttfamily\tiny,
      keywordstyle=\ttfamily\textcolor{blue},
      stringstyle=\ttfamily\textcolor{red},
      commentstyle=\ttfamily\textcolor{green},
      breaklines=true,
      extendedchars=true,
      basicstyle=\ttfamily\tiny,
      showstringspaces=false
    }
%%%%%%%%%%%%%%%%%%%%
\title{\texttt{LINGI 1122}: Projet « Machine de Turing » \\ {\large Groupe 13B: Rapport 2}}
\author{Matthieu \textsc{Baerts} \and Pieter \textsc{Hollevoet} \and Hélène \textsc{Verhaeghe}}
    \date{\today}
\begin{document}
\maketitle
% \tableofcontents


\section*{Introduction}
Nous n'avons pas rencontré énormément de problèmes pour implémenter le squelette qui nous était donné. Les seuls points que nous jugeons utiles de mentionner sont en rapport avec des variables d'état que nous n'avons pas utilisées car en implémentant les différentes méthodes, elles se sont jugées inutiles ou redondantes :

\section{Analyse}
\subsection{\texttt{repOkNb} et \texttt{checkedFor}}
Ces deux variables d'instance respectivement des classes \texttt{TapeA\_B} et \texttt{Cell} ne nous étaient pas spécifiées clairement.\\
Apparemment, elles devaient servir à la méthode \texttt{repOk}\footnote{Après la remise de la première version du rapport, il s'est avéré que, même si cette méthode se trouvait dans le squelette du programme donné, cette méthode ne devait pas être implémentée.} mais nous ne leur avons trouvé aucune utilité. Comme nous vérifions au fur et à mesure de bien lier chaque cellule dans les deux sens, nous ne devons pas les marquer mais simplement vérifier que nous ne retombons pas sur le premier maillon.\\
Ainsi dans le cas d'une boucle qui arriverait sur un des maillons déjà testé, en testant le maillon précédent, nous savons voir que l'on ne revient pas sur nos pas et donc qu'il y a une boucle quelque part.

\subsection{\texttt{NullPreviousEncountered} et \texttt{NullNextEncountered}}
Pour ces deux variables, nous savions à quoi elles étaient utiles : savoir si la cellule précédente (resp. suivante) existait ou non. Maintenant, nous ne voyons pas de justification à leur existence vu qu'il était facilement possible de les vérifier grâce à la structure de la cellule: \texttt{cell.previous == null}.\\
Vu que ces variables n'apportaient que de la redondance et donc un risque d'erreur supplémentaire, nous ne les avons pas utilisées.

\section*{Conclusion}
Nous avons complété le squelette comme demandé mais nous avons préféré corrigé deux-trois petites choses comme cité dans l'analyse.

\appendix
\section{Implémentation}
\lstinputlisting{TapeA_B.java}

\end{document}
