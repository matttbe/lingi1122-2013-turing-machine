\documentclass[a4paper,11pt]{article}
\usepackage[T1]{fontenc}
\usepackage[utf8]{inputenc}
\usepackage{lmodern}
\usepackage[francais]{babel}
\usepackage{amsmath} % math
\usepackage{amssymb} % math
\usepackage{gensymb} % math
\usepackage{graphicx} % images
% \usepackage{qtree}    % dessiner des arbres %% => texlive-humanities
\usepackage{url}
\urlstyle{sf}
\usepackage[usenames]{color}
\usepackage[french]{varioref} % \vpageref
\usepackage[top=2.5cm, bottom=2.5cm, left=2.5cm, right=2.5cm]{geometry}
\definecolor{codeBlue}{rgb}{0,0,1}
\definecolor{webred}{rgb}{0.5,0,0}
\definecolor{codeGreen}{rgb}{0,0.5,0}
\definecolor{codeGrey}{rgb}{0.6,0.6,0.6}
\definecolor{webdarkblue}{rgb}{0,0,0.4}
\definecolor{webgreen}{rgb}{0,0.3,0}
\definecolor{webblue}{rgb}{0,0,0.8}
\definecolor{orange}{rgb}{0.7,0.1,0.1}
\usepackage{caption}
\renewcommand{\familydefault}{\sfdefault}
\usepackage{listings}        % Pour l'insersion de fichiers de codes sources.
\lstset{
      language=bash,
      flexiblecolumns=true,
      numbers=left,
      stepnumber=1,
      numberstyle=\ttfamily\tiny,
      keywordstyle=\ttfamily\textcolor{blue},
      stringstyle=\ttfamily\textcolor{red},
      commentstyle=\ttfamily\textcolor{green},
      breaklines=true,
      extendedchars=true,
      basicstyle=\ttfamily\scriptsize,
      showstringspaces=false
    }
%%%%%%%%%%%%%%%%%%%%
\title{\texttt{LINGI 1122}: Projet « Machine de Turing » \\ {\large Groupe 13B: Rapport 1}}
\author{Matthieu \textsc{Baerts} \and Pieter \textsc{Hollevoet} \and Hélène \textsc{Verhaeghe}}
    \date{\today}
\begin{document}
\maketitle
% \tableofcontents
\section{Sous-Machines}
\subsection{\texttt{findFirstBlankOnTheLeft()} (\texttt{F1BL})}
\paragraph{Pré:} La tête de lecture se trouve sur un caractère quelconque du ruban.
\begin{verbatim}
-------------------------------------------------v---------------------
 suite de caractères|B|suite de caractères non B|C|suite de caractères
-----------------------------------------------------------------------
\end{verbatim}
où C est un caractère quelconque
\paragraph{Post:} La tête de lecture se trouve sur le premier blanc à gauche de l'ancien emplacement de la tête de lecture. Le ruban est inchangé.
\begin{verbatim}
---------------------v-------------------------------------------------
 suite de caractères|B|suite de caractères non B|C|suite de caractères
-----------------------------------------------------------------------
\end{verbatim}
\subsection{\texttt{findFirstBlankOnTheRight()} (\texttt{F1BR})}
\paragraph{Pré:} La tête de lecture se trouve sur un caractère quelconque du ruban.
\begin{verbatim}
---------------------v-------------------------------------------------
 suite de caractères|C|suite de caractères non B|B|suite de caractères
-----------------------------------------------------------------------
\end{verbatim}
où C est un caractère quelconque
\paragraph{Post:} La tête de lecture se trouve sur le premier blanc à droite de l'ancien emplacement de la tête de lecture. Le ruban est inchangé.
\begin{verbatim}
-------------------------------------------------v---------------------
 suite de caractères|C|suite de caractères non B|B|suite de caractères
-----------------------------------------------------------------------
\end{verbatim}
où C est un caractère quelconque
\subsection{\texttt{convertToUnaryLeft()}}
\paragraph{Pré:} La tête de lecture se trouve sur un blanc et, à sa gauche, se trouve une suite valide en binaire délimitée par un blanc.
\begin{verbatim}
---------------------------------------v---------------------
 suite de caractères|B|suite de 0 et 1|B|suite de caractères
-------------------------------------------------------------
\end{verbatim}
\paragraph{Post:} La tête de lecture est de retour sur le même blanc. La suite binaire a été convertie en une suite unaire délimitée par un blanc et se trouve à gauche de la tête de lecture.
La taille de cette suite peut être différente de la précédente, les changements se font vers la gauche, c'est-à-dire que le blanc à gauche peut avoir été décalé. La séquence de caractères à droite de la tête de lecture reste inchangée.
\begin{verbatim}
----------------------------------v-------------------------------
 suite de caractères|B|suite de 1|B|suite de caractères inchangés
------------------------------------------------------------------
\end{verbatim}
\subsection{\texttt{convertToUnaryRight()}}
\paragraph{Pré:}
La tête de lecture se trouve sur un blanc et, à sa droite, se trouve une suite valide en binaire délimitée par un blanc.
\begin{verbatim}
---------------------v---------------------------------------
 suite de caractères|B|suite de 0 et 1|B|suite de caractères
-------------------------------------------------------------
\end{verbatim}
\paragraph{Post:}
La tête de lecture est de retour sur le même blanc. La suite binaire a été convertie en une suite unaire délimitée par un blanc et se trouve à droite de la tête de lecture.
La taille de cette suite peut être différente de la précédente, les changements se font vers la droite, c'est-à-dire que le blanc à droite peut avoir été décalé. La séquence de caractères à gauche de la tête de lecture reste inchangée.
\begin{verbatim}
-------------------------------v----------------------------------
 suite de caractères inchangés|B|suite de 1|B|suite de caractères
------------------------------------------------------------------
\end{verbatim}
\subsection{\texttt{convertToBinaryLeft()}}
\paragraph{Pré:}
La tête de lecture se trouve sur un blanc et, à sa gauche, se trouve une suite valide en unaire délimitée par un blanc.
\begin{verbatim}
----------------------------------v---------------------
 suite de caractères|B|suite de 1|B|suite de caractères
--------------------------------------------------------
\end{verbatim}
\paragraph{Post:}
La tête de lecture est de retour sur le même blanc. La suite unaire a été convertie en une suite binaire délimitée par un blanc et se trouve à gauche de la tête de lecture.
La taille de cette suite peut être différente de la précédente, les changements se font du coté gauche, c'est-à-dire que le blanc à gauche peut avoir été décalé. La séquence de caractères à droite de la tête de lecture reste inchangée.
\begin{verbatim}
-------------------------------------------v-------------------------------
 suite de caractères |B|suite de 0 ou de 1|B|suite de caractères inchangée
---------------------------------------------------------------------------
\end{verbatim}
\subsection{\texttt{convertToBinaryRight()}}
\paragraph{Pré:}
La tête de lecture se trouve sur un blanc et, à sa droite, se trouve une suite valide en unaire délimitée par un blanc.
\begin{verbatim}
---------------------v----------------------------------
 suite de caractères|B|suite de 1|B|suite de caractères
--------------------------------------------------------
\end{verbatim}
\paragraph{Post:}
La tête de lecture est de retour sur le même blanc. La suite unaire a été convertie en une suite binaire délimitée par un blanc et se trouve à droite de la tête de lecture.
La taille de cette suite peut être différente de la précédente, les changements se font du coté droit, c'est-à-dire que le blanc à droite peut avoir été décalé. La séquence de caractères à gauche de la tête de lecture reste inchangée.
\begin{verbatim}
--------------------------------v------------------------------------------
 suite de caractères inchangée |B|suite de 0 ou de 1|B|suite de caractères
---------------------------------------------------------------------------
\end{verbatim}
\subsection{\texttt{copyLeftOfLeftSequence()}}
\paragraph{Pré:} 
\begin{verbatim}
---------------------v----------------------------------
 suite de blanc|suite de caractères|B|suite de caractère|B|suite de caractères
--------------------------------------------------------
\end{verbatim}
\paragraph{Post:}
\begin{verbatim}
\end{verbatim}
\subsection{\texttt{copyAfterNextFirstBlank()}}
\paragraph{Pré:}
\begin{verbatim}
\end{verbatim}
\paragraph{Post:}
\begin{verbatim}
\end{verbatim}
\subsection{\texttt{copyAfterNextSecondBlank()}}
\paragraph{Pré:}
\begin{verbatim}
\end{verbatim}
\paragraph{Post:}
\begin{verbatim}
\end{verbatim}
\section{Addition}\label{add}
\begin{verbatim}
---------------------------------------------------------v---------------------
 suite de caractères|B|suite de 0 et 1|B|suite de 0 et 1|B|suite de caractères
-------------------------------------------------------------------------------
\end{verbatim}
Premièrement, on place la tête de lecture entre les deux nombres grâce à \texttt{findFirstBlankOnTheLeft()}.
\begin{verbatim}
---------------------------------------v---------------------------------------
 suite de caractères|B|suite de 0 et 1|B|suite de 0 et 1|B|suite de caractères
-------------------------------------------------------------------------------
\end{verbatim}
Ensuite, on utilise \texttt{convertToUnaryRight()} et \texttt{convertToUnaryLeft()} pour transformer les deux nombres en unaire.Le blanc entre les deux nombres est conservé.
\begin{verbatim}
----------------------------------v------------------------------------
 suite de caractères|B|suite de 1|B|suite de n 1|B|suite de caractères
-----------------------------------------------------------------------
\end{verbatim}
Pour réaliser l'addition en elle-même, il ne reste plus qu'à retirer un bâton à l'extrême droite du nombre de droite (utiliser \texttt{findFirstBlankOnTheRight()}) et à le placer à la place du blanc séparent initialement les deux nombres.
\begin{verbatim}
----------------------------------v--------------------------------------
 suite de caractères|B|suite de 1|1|suite de n-1 1|B|suite de caractères
-------------------------------------------------------------------------
\end{verbatim}
Pour finir, se replacer à l'extrême droite du nombre final et appliquer \texttt{convertToBinaryLeft()}. La tête de lecture sera déjà au bon endroit.
\begin{verbatim}
---------------------------------------v---------------------
 suite de caractères|B|suite de 0 et 1|B|suite de caractères
-------------------------------------------------------------
\end{verbatim}
\section{Soustraction}
Premièrement, on place la tête de lecture entre les deux nombres grâce à \texttt{findFirstBlankOnTheLeft()}.
Ensuite, on utilise \texttt{convertToUnaryRight()} et \texttt{convertToUnaryLeft()} pour transformer les deux nombres en unaire. Le blanc entre les deux nombres est conservé tout comme les deux premières étapes de la section \ref{add}. Un exemple ici la soustraction de 5 par 3.
\begin{verbatim}
-----------------------------v---------------------------
 suite de caractères|B|11111|B|111|B|suite de caractères
---------------------------------------------------------
\end{verbatim}
Pour réaliser la soustraction en elle-même, il ne reste plus qu'à retirer un bâton (en le remplaçant par un blanc) dans la suite de droite en partant de la droite de cette suite : c'est-à-dire en utilisant \texttt{findFirstBlankOnTheRight()} puis en se plaçant à gauche, en regardant si le caractère est bien à 1 (sinon la soustraction est terminée) avant de remplacer ce caractère par un blanc pour enfin revenir au blanc situé entre les deux nombres avec \texttt{findFirstBlankOnTheLeft()}.\\
\begin{verbatim}
-----------------------------v---------------------------
 suite de caractères|B|11111|B|11|BB|suite de caractères
---------------------------------------------------------
\end{verbatim}
On fait pareil dans la suite de gauche en partant de la gauche de cette suite : c'est-à-dire en utilisant \texttt{findFirstBlankOnTheLeft()} puis en se plaçant à droite, en regardant si le caractère  est bien à 1 (sinon la soustraction vaut 0, c'est terminé) avant de remplacer ce caractère par un blanc pour enfin revenir au blanc situé entre les deux nombres avec \texttt{findFirstBlankOnTheRight()}.\\
\begin{verbatim}
-----------------------------v---------------------------
 suite de caractères|BB|1111|B|11|BB|suite de caractères
---------------------------------------------------------
\end{verbatim}
Ces deux opérations sont donc à réaliser tant que l'une des deux suites ne sont pas (2 blancs consécutifs):
\begin{verbatim}
-----------------------------v---------------------------
 suite de caractères|BBBB|11|B||BBBB|suite de caractères
---------------------------------------------------------
\end{verbatim}
Pour finir, il ne reste plus qu'à appliquer \texttt{convertToBinaryLeft()}. La tête de lecture sera déjà au bon endroit.
\begin{verbatim}
-----------------------------v---------------------------
 suite de caractères|BBBB|10|B||BBBB|suite de caractères
---------------------------------------------------------
\end{verbatim}
\section{Multiplication}
Premièrement, on place la tête de lecture entre les deux nombres grâce à \texttt{findFirstBlankOnTheLeft()}.
Ensuite, on utilise \texttt{convertToUnaryRight()} et \texttt{convertToUnaryLeft()} pour transformer les deux nombres en unaire. Le blanc entre les deux nombres est conservé tout comme les deux premières étapes de la section \ref{add}.
\begin{verbatim}
-----------------------v---------------------------
 suite de B|suite de 1|B|suite de n 1|B|suite de B
---------------------------------------------------
\end{verbatim}
\section{Division}
\end{document}
